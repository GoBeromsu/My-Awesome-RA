\section{Related work}
\label{sc:related-work}
\subsection{Commit-Related Tasks}

Commit-level research is commonly grouped into three areas: \emph{generation}, \emph{classification}, and \emph{refinement}.
Generation tasks include commit message generation (CMG), code change summarization, code review generation (CRG), and just-in-time comment updates (JITCU). 
Classification typically targets the recovery of semantic intent, often via CCS types. 
Refinement leverages commit context for code improvement and defect prediction.
Comparative studies report that LLMs achieve strong performance across CMG, CRG, and JITCU under in-context learning (ICL) and PEFT, albeit with substantial computational and latency costs \citep{Fan2024Exploring}.
Fine-tuned SLMs trained on high-quality data demonstrate competitive performance and offer a cost-efficient alternative \citep{Li2024Understanding}.
Code-change-oriented pretraining further improves generation quality.
For example, CCT5, pre-trained on \emph{CodeChangeNet} (1.5M commit–message pairs), consistently outperforms earlier approaches \citep{Lin2023CCT5}.
Refinement tasks extend these gains by exploiting commit context for automated refactoring and just-in-time defect prediction \citep{Li2024Understanding, Lin2023CCT5}.
However, most CCS-based classification assumes a single dominant intent per commit, leaving multi-intent (tangled) commits underexplored and motivating our multi-label formulation.

\subsection{Optimizing SLMs for Code-level Tasks}
Although our methodology achieves strong baseline performance using a curated dataset and CoT prompting, its capabilities can be further enhanced by incorporating techniques from recent SLM research.
To mitigate the limitation that language models often identify what changed but fail to infer why, Retrieval-Augmented Generation (RAG) can supply explicit, relevant examples during inference \citep{Li2024Understanding}.
More sophisticated prompt-engineering strategies also offer promising directions for improving reasoning on complex code changes \citep{Hou2024Large}.
Recent studies further suggest that SLM reasoning ability can be enhanced through distillation.
Although CoT prompting elicits reasoning in LLMs, such capabilities typically emerge only in very large models.
\citet{Ranaldi2024Aligning} show that instruction-tuned student models trained on a small number of high-quality distilled CoT demonstrations can recover much of this reasoning ability at smaller scales.
In particular, CoT-based demonstrations are most effective for reasoning alignment, especially when derived from in-family teacher models.
These findings indicate that SLMs, despite their parameter constraints, can acquire multi-step reasoning skills through carefully curated distillation rather than prompting alone.
